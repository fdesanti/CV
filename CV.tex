\documentclass[a4paper]{moderncv}

\usepackage{graphicx}
\usepackage{amssymb}
\usepackage{amsmath}
\usepackage[utf8]{inputenc}
\usepackage{longtable}
\usepackage{xcolor}
\usepackage{xspace}
\newcommand{\rsquo}{{\tt\char'023}}
%\usepackage{orcidlink}
%\newcommand{\orcid}[1]{\href{https://orcid.org/#1}{\textcolor{lime}{\aiOrcid}}}

\moderncvstyle{banking}
\moderncvcolor{blue}

\usepackage[sfdefault,lf]{carlito}
\usepackage[T1]{fontenc}
\renewcommand*\oldstylenums[1]{\carlitoOsF #1}

\usepackage[top=2cm,bottom=2cm,left=2cm,right=2cm,bindingoffset=0cm]{geometry}
\setlength{\hintscolumnwidth}{3cm}
\usepackage{enumitem}
\setlist{nolistsep}

\makeatletter
\renewcommand*{\bibliographyitemlabel}{\@biblabel{\arabic{enumiv}}}
\makeatother

\newcommand{\mnras}{Monthly Notices of the Royal Astronomical Society\xspace}
\newcommand{\mnrasl}{Monthly Notices of the Royal Astronomical Society Letters\xspace}
\newcommand{\prd}{Physical Review D\xspace}
\newcommand{\prdl}{Physical Review D Letters\xspace}
\newcommand{\prdrc}{Physical Review D Rapid Communications\xspace}
\newcommand{\prl}{\textbf{Physical Review Letters}\xspace}
\newcommand{\prlplain}{{Physical Review Letters\xspace}}
\newcommand{\cqg}{Classical and Quantum Gravity\xspace}
\newcommand{\aap}{Astronomy \& Astrophysics\xspace}
\newcommand{\prr}{Physical Review Research\xspace}
\newcommand{\apj}{Astrophysical Journal\xspace}
\newcommand{\apjl}{Astrophysical Journal Letters\xspace}
\newcommand{\grg}{General Relativity and Gravitation\xspace}
\newcommand{\natastro}{Nature Astronomy\xspace}
\newcommand{\lrr}{Living Reviews in Relativity\xspace}

\long\def\suppress#1\endsuppress{%
  \begingroup%
    \tracinglostchars=0%
    \let\selectfont=\nullfont
    \nullfont #1\endgroup}

\fancypagestyle{headonly}{
\fancyfoot{}
\fancyfoot[r]{\textcolor{color2}{\thepage}}
\fancyhead{}
}

\newcommand{\mytitle}[1]{\title{#1\vspace{0.15cm}}}


\firstname{Federico}
\familyname{De Santi}

\extrainfo{\normalsize f.desanti@campus.unimib.it $\;\;\bullet\;\;$  $\;\;\bullet\;\;$ \today}

\mytitle{Curriculum Vitae}
\definecolor{color1}{rgb}{0.0, 0.45, 0.81}
\definecolor{mark_color}{rgb}{0, 0, 0}

\begin{document}
\pagestyle{headonly}

\makecvtitle

%\cvitem{}{\emph{\vspace{-1cm}\\
%$\quad$ Relativistic astrophysicist, studying the impact of Einstein's general relativity on the astrophysical world. Research interests include gravitational-wave astronomy, black-hole binary dynamics, Bayesian statistics, and machine-learning applications.}}
\cvitem{}{\emph{\vspace{-1cm}\\
$\quad$ PhD student in Physics and Astronomy at the University of Milano-Bicocca focusing on Gravitational Waves Astrophysics and Data Analysis. Research interests include eccentric compact binaries, Bayesian Statistics and (Probabilistic) Machine Learning applications.}}


\section{Contact}

\cvitem{Email}{\href{mailto:f.desanti@campus.unimib.it}{f.desanti@campus.unimib.it}, \href{mailto:f.desanti@studenti.unipi.it}{f.desanti@studenti.unipi.it}}
\cvitem{Address}{Universit\`{a} degli Studi di Milano-Bicocca, Piazza della Scienza 3, 20126 Milano, Italy.}
%\cvitem{Website}{\href{https://www.davidegerosa.com/}{www.davidegerosa.com}}

%\cvitem{Citizenship}{Italy, EU.}


\section{Education}

\cventry{2024-current}{Ph.D. in Physics and Astronomy}{University of Milano-Bicocca}{Milan, Italy}{}{}
\vspace{-0.1cm}
\begin{tabular}{rcl}
&\hspace{0.4cm} &$\circ\;\;${\textit{Supervisor}}: D. Gerosa.
\\
\end{tabular}

\vspace{0.2cm}
\cventry{2019-2023}{Master's degree in Physics}{\newline University of Pisa}{Pisa, Italy}{}{}
\vspace{-0.1cm}
\begin{tabular}{rcl}
&\hspace{0.4cm} &$\circ\;\;${\textbf{\textit{Final degree grade}}}: 110/110.\\
&\hspace{0.4cm} &$\circ\;\;${\textbf{\textit{Thesis}}}: \textit{Gravitational Waves from Binary Close Encounters: Fast Parameter Estimation} \\&\hspace{0.4cm} & 
\hspace{0.4cm}
\textit{\phantom{Thesis } with Normalizing Flows}.\\
&\hspace{0.4cm} &$\circ\;\;${\textbf{\textit{Thesis advisors}}}: M. Razzano, F. Fidecaro. Thesis resulted in 1 publications.\\

\end{tabular}

\vspace{0.2cm}
\cventry{2015-2019}{Bachelor's degree in Physics}{University of Pisa}{Pisa, Italy}{}{}
\vspace{-0.1cm}
\begin{tabular}{rcl}
&\hspace{0.4cm} &$\circ\;\;${\textbf{\textit{Final degree grade}}}: 92/110.\\
&\hspace{0.4cm} &$\circ\;\;${\textbf{\textit{Thesis}}}: \textit{Bell's Inequalities}.\\
&\hspace{0.4cm} &$\circ\;\;${\textbf{\textit{Thesis advisor}}}:  C. Bonati.\\
\end{tabular}


\vspace{0.2cm}
\cventry{2010-2015}{Scientific High School Diploma}{\newline Liceo Scientifico A. Vallisneri}{Lucca, Italy}{}{}  %
\vspace{-0.1cm}
\begin{tabular}{rcl}
&\hspace{0.4cm} &$\circ\;\;${\textbf{\textit{Final grade}}}:  100/100.\\
\end{tabular}

%\vspace{0.2cm}

\section{Formation}

\cventry{Feb. 2024 - Nov. 24}{Research Scholarship (Borsa di Studio e Approfondimento), Department of Physics}{Universit\`{a} di Pisa}{Pisa, Italy}{}{}
\vspace{-0.1cm}
\begin{tabular}{rcl}
&\hspace{0.4cm} &$\circ\;\;${\textbf{\textit{Project title}}}: \textit{Machine Learning methods for the detection of gravitational wave signals}\\&\hspace{0.4cm} &
\textit{\phantom{Project title    }produced by close encounters between compact objects in binary systems}\\
&\hspace{0.4cm} &$\circ\;\;${\textbf{\textit{Supervisor}}}:  M. Razzano.\\
\end{tabular}
\vspace{0.2cm}


\section{Metrics}

\cvitem{}{\begin{tabular}{rcl}
\textcolor{mark_color}{\textbf{Publications}}: &\hspace{0.3cm} &
\textbf{1} papers published in major peer-reviewed journals,
\\ & &
\textbf{1} other publications (white papers, long-authorlist reviews, proceedings, software, etc)
\\ & &
(out of which \textbf{1} first-authored papers and
\textbf{0} papers covered by press releases).
\end{tabular} }
Summary metrics reported using ADS and InSpire excluding [including] long-authorlist papers:
\\
\textcolor{mark_color}{\textbf{Total number of citations}}: 1 [1].
 --- 
\textcolor{mark_color}{\textbf{h-index}}: 1 [1].
\\
\textcolor{mark_color}{\textbf{Web links to list services}}:
\href{https://ui.adsabs.harvard.edu/search/q=orcid%3A0009-0000-2445-5729&sort=date+desc}{\textsc{ADS}};
\href{https://inspirehep.net/literature?sort=mostrecent&size=25&page=1&q=a%20F.De.Santi.2}{\textsc{InSpire}};
\href{https://arxiv.org/a/desanti_f_1.html}{\textsc{arXiv}};
\href{https://orcid.org/0009-0000-2445-5729}{\textsc{ORCID}}.

%\textbf{Full list of publications} available 
%mark_CVshort
%below and
%mark_CVshort
%at \href{http://www.davidegerosa.com/pub}{\texttt{www.davidegerosa.com/pub}}.

\vspace{0.2cm}

\cvitem{}{\begin{tabular}{rcl}
\textcolor{mark_color}{\textbf{Presentations}}: &\hspace{0.3cm} &
\textbf{0} talks at conferences,
\textbf{0} talks at department seminars,
\textbf{0} posters at conferences,
\\ & &
(out of which \textbf{0.0} invited presentations),
\textbf{0} lecture at PhD schools,
\textbf{0} outreach talks.
\end{tabular} }

%\textbf{Full list of presentations} available
%mark_CVshort
%below and
%mark_CVshort
%at \href{http://www.davidegerosa.com/talks}{\texttt{www.davidegerosa.com/talks}}.



%%%%%%%
\vspace{0.2cm}


%\cventry{2014-2015}{Darwin College}{Travel fund}{Cambridge, UK}{University of Cambridge}{}
%\cventry{2015}{Cambridge Philosophical Society}{Travel fund}{Cambridge, UK}{University of Cambridge}{}
%\cvitem{Nov 2011}{Award assigned by the cultural association "Famiglia Legnanese" to  best Italian College students, Legnano MI, Italy.}
%\cvitem{Jul 2007}{Award assigned by the website \textit{matematicamente.it} to the best italian high-school research project, presenting my work  {"Do I Dare disturb the Universe?"} about  evidences for dark matter.}



\vspace{0.2cm}
%\newpage{}
%\vspace{-0.1cm}

\vspace{0.2cm}
%\textbf{\textcolor{black}{Organized conferences}}\vspace{0.05cm}\\
%
%\cvitemwithcomment{}{\hspace{0.4cm}$\circ\;$ XXVI SIGRAV conference on general relativity and gravitation}{2025}\vspace{-0.1cm}
%\cvitemwithcomment{}{\hspace{0.4cm}$\phantom{\circ}\;$ University of Milano-Bicocca, Italy. %\href{https://sites.google.com/unimib.it/gwsnowballs}{\texttt{[link]}}
%}{} \vspace{-0.1cm}


%$\circ\;$ GWIC 3G Science case subcommittee, & $\circ\;$ Royal Astronomical Society,\\

\vspace{0.2cm}
\textbf{\textcolor{black}{Memberships}}\vspace{0.05cm}\\
\cvitemwithcomment{}{\hspace{0.4cm}$\circ\;$ TEONGRAV National Initiative (Gravity Theory), (INFN).}{2024-now}\vspace{-0.1cm}
\cvitemwithcomment{}{\hspace{0.4cm}$\circ\;$ Italian National Institute for Nuclear Physics (INFN)}{2023-now}\vspace{-0.1cm}
\cvitemwithcomment{}{\hspace{0.4cm}$\circ\;$ Einstein Telescope Scientific Collaboration.}{2024}\vspace{-0.1cm}
\cvitemwithcomment{}{\hspace{0.4cm}$\circ\;$ Virgo Collaboration: \textit{(part of the LIGO-Virgo-Kagra (LVK) Collaboration)}.}{2023-2024}\vspace{-0.1cm}
\cvitemwithcomment{}{\hspace{0.4cm}$\circ\;$ American Physical Society (APS).}{2016-2023}\vspace{-0.1cm}
\cvitemwithcomment{}{\hspace{0.4cm}$\circ\;$ Associazione Italiana Studenti di Fisica (AISF).}{2016-2019}\vspace{-0.1cm}



\section{Skills}

\cvitem{Programming languages}{Python (advanced), Matlab, Mathematica, C (basic), LaTeX.}
\cvitem{Other scientific tools}{PyTorch, TensorFlow, LIGO lalsuite, GwPy, PyCBC, Bilby, Version System Control with Git, cloud computing.}
\cvitem{Data Analysis:}{Bayesian Statistics \& Inference, Probability Theory, (Probabilistic) Machine Learning, Normalizing Flows.}
\cvitem{Languages}{English (fluent C1), Italian (native).}


\section{Research Interests}
\textbf{\textcolor{black}{Gravitational Wave Physics}}
\begin{itemize}[label=-, leftmargin=0.5cm]
    \item Data analysis and Bayesian parameter estimation
    \item Astrophysics and populations of (eccentric) coalescing compact binaries
    \item Inference and signal recovery in Third Generation Detectors
    \item Cosmology with gravitational waves
    \item Multimessenger astronomy
\end{itemize}
\medskip
\textbf{\textcolor{black}{Machine Learning}}
\begin{itemize}[label=-, leftmargin=0.5cm]
    \item \textit{Deep Generative Models:} Normalizing Flows, Variational Autoencoders
    \item Probabilistic Machine Learning
    \item Machine learning applications to gravitational-wave data
    \item Transformers
\end{itemize}

\section{Summer Schools}
\cventry{28-31 Mar. 2023}{}{\href{https://indico.physics.auth.gr/event/14/}{4th G2Net Training School}}{Thessaloniki, Greece}{}{}
\begin{tabular}{rcl}
    &\hspace{0.4cm} &$\circ\;\;$School devoted to ML applications in Gravitational Wave Astronomy\\
    &\hspace{0.4cm} &$\circ\;\;$Among best results in the \href{https://github.com/niksterg/g2net_4th_training_school_thessaloniki_2023}{\textit{Hackathon}}\\
\end{tabular}
\vspace{0.2cm}


\section{Certifications}
\cventry{2015}{Cambridge English}{IELTS (Academic)}{}{}{\textbf{Overall score}: 7.0 (\textbf{C1} equivalent)}
\vspace{0.2cm}
\cventry{2013}{Cambridge ESOL (entry 3)}{PET}{}{}{\textit{"Passed with merit"} Level B1}
\vspace{0.2cm}
\cventry{2014}{WorldScienceU}{Special Relativity Course}{}{}{Special Relativity course hosted by Brian Greene at World Science U (Columbia University)}







%\section{Hobbies}

%mark_CVshort

\pagebreak
\section{Full publication list}\vspace{0.2cm} 

\phantom{phantom text}

\cvitem{}{\small\hspace{-1cm}\begin{longtable}{rp{0.3cm}p{15.8cm}}
%
\end{longtable} }
\textcolor{color1}{\textbf{Papers in major peer-reviewed journals:}}
\phantom{phantom text}

\cvitem{}{\small\hspace{-1cm}\begin{longtable}{rp{0.3cm}p{15.8cm}}
%
\textbf{1.} & & \textit{Deep learning to detect gravitational waves from binary close encounters: Fast parameter estimation using normalizing flows.}
\newline{}
\textbf{F. De Santi}, M. Razzano, F. Fidecaro, L. Muccillo, L. Papalini, B. Patricelli.
\newline{}
\href{https://journals.aps.org/prd/abstract/10.1103/PhysRevD.109.102004}{\prd 109 (2024) 102004}. \href{https://arxiv.org/abs/2404.12028}{arXiv:2404.12028 [gr-qc].}
\vspace{0.09cm}\\
%
\end{longtable} }
\textcolor{color1}{\textbf{Conference proceedings:}}
\phantom{phantom text}

\cvitem{}{\small\hspace{-1cm}\begin{longtable}{rp{0.3cm}p{15.8cm}}
%
\textbf{1.} & & \textit{Seismic isolation systems for next-generation gravitational wave detectors.}
\newline{}
M. Razzano, F. Spada, G. Balestri, A. Basti, L. Bellizzi, \textbf{F. De Santi}, F. Fidecaro, A. Fiori, F. Frasconi, A. Gennai, L. Lucchesi, L. Muccillo, L. Orsini, M. Antonietta Palaia, L. Papalini, F. Pilo, P. Prosperi, M. Vacatello.
\newline{}
\href{https://www.sciencedirect.com/science/article/pii/S0168900224006016}{Nuclear Instruments and Methods in Physics Research Section A: Accelerators, Spectrometers, Detectors and Associated Equipment}. 
\vspace{0.09cm}\\
%
\end{longtable} }


\section{Full presentation list}\vspace{0.2cm} 

Invited talks marked with *.
\vspace{0.2cm}

\textcolor{color1}{\textbf{Talks at conferences:}}
\vspace{-0.5cm}

\cvitem{}{\small\hspace{-1cm}\begin{longtable}{rp{0.3cm}p{15.8cm}}
%
\textbf{4.} &  & \textbf{Deep learning to detect gravitational waves from binary close encounters: Fast parameter estimation with normalizing flows.}
\newline{}
\textit{LIGO-Virgo-KAGRA (LVK) Collaboration Meeting (Baton Rouge, USA)}, 11-14 Mar 2024.
\newline{}
\textcolor{color1}{$\bullet$} (online).
\vspace{0.05cm}\\
%
\textbf{3.} &  & \textbf{Deep learning to detect gravitational waves from binary close encounters: Fast parameter estimation with normalizing flows.}
\newline{}
\textit{\href{https://indico.ego-gw.it/event/666/overview}{Integrated Activities for the High Energy Astrophysics Domain (AHEAD2020) Workshop} - European Gravitational Wave Observatory (EGO), Cascina}, 11-14 Mar 2024.
\vspace{0.05cm}\\
%
\textbf{2.} &  & \textbf{GW from Binary Close Encounters: Analysis with Normalizing Flow.}
\newline{}
\textit{Virgo Week - European Gravitational Wave Observatory (EGO), Cascina}, 11-14 Mar 2024.
\vspace{0.05cm}\\
%
\textbf{1.} &  & \textbf{GW from Binary Close Encounters: Analysis with Normalizing Flow.}
\newline{}
\textit{Burst (LVK) Collaboration Meeting}, 11-14 Mar 2024.
\newline{}
\textcolor{color1}{$\bullet$} online.
\vspace{0.05cm}\\
%
\end{longtable} }
\textcolor{color1}{\textbf{Talks at department seminars:}}
\vspace{-0.5cm}

\cvitem{}{\small\hspace{-1cm}\begin{longtable}{rp{0.3cm}p{15.8cm}}
%
\textbf{1.} & * & \textbf{Probabilistic Machine Learning for the study of burst sources.}
\newline{}
\textit{Virgo Pisa Internal Workshop (INFN Pisa)}, 22-23 May 2024.
\vspace{0.05cm}\\
%
\end{longtable} }
\textcolor{color1}{\textbf{Posters at conferences:}}
\vspace{-0.5cm}

\cvitem{}{\small\hspace{-1cm}\begin{longtable}{rp{0.3cm}p{15.8cm}}
%
\textbf{3.} &  & \textbf{HYPERION: a Normalizing Flow based pipeline for the rapid parameter estimation of eccentric Close Encounters.}
\newline{}
\textit{\href{https://agenda.infn.it/event/40101/overview}{GraSP24 - Gravity Shape Pisa 2024 Conference}}, 23-25 Oct 2024.
\vspace{0.05cm}\\
%
\textbf{2.} &  & \textbf{HYPERION: a Normalizing Flow based pipeline for the rapid parameter estimation of eccentric Close Encounters.}
\newline{}
\textit{\href{https://agenda.infn.it/event/38056/overview}{V GraviGammaNu Workshop 2024} (Bari)}, 9-11 Oct 2024.
\vspace{0.05cm}\\
%
\textbf{1.} &  & \textbf{Gravitational Waves from Binary Close Encounters: Fast Parameter Estimation with Normalizing Flows.}
\newline{}
\textit{LIGO-Virgo-KAGRA (LVK) Collaboration Meeting (Toyama, Japan)}, 11-14 Sep 2023.
\vspace{0.05cm}\\
%
\end{longtable} }
%mark_CVshort

\pagebreak
\section{Research activity}
\textbf{\textcolor{black}{\hspace{0.4cm}$\circ\;$ Detection and Parameter Estimation of eccentric gravitational wave signals}}\vspace{0.05cm}\\
\indent There is evidence for the presence of many astrophysical formation channels within the
population of compact fusion objects we observe today.
Eccentricity is one of the key aspects in distinguishing and studying the sources of
gravitational waves and their astrophysical origin.\\
During my Master's thesis, I began research work in this area, focusing on the study of highly
eccentric close encounters ($e\sim1$) in dense stellar environments: an extremely challenging and
as yet undetected GW source type.
I developed a fast inference pipeline (\textsc{HYPERION}) based on Normalizing Flows and ML,
capable of performing accurate detection and full Bayesian parameter estimation several
orders of magnitude faster than traditional methods.\\
My work, already presented to the LIGO-Virgo-KAGRA Collaboration, has furthermore led to a
peer-reviewed publication as a first author in Physical Review D.
\medskip

\textbf{\textcolor{black}{\hspace{0.4cm}$\circ\;$ ML applied to the Einstein Telescope Mock Data Challenge}}\vspace{0.05cm}\\
\indent Due to the enormous leap in sensitivity, the analysis of the Einstein Telescope data is
expected to face several challenges: e.g. the presence of multiple overlapped signals.\\
The ET Collaboration has therefore launched a Mock Data Challenge to test the capabilities of
various pipelines in signal extraction and parameter estimation. The results of this challenge
will provide crucial insights into the scientific potential of the Einstein Telescope.\\
During the period of my research fellowship, together with L. Papalini (PhD student in Pisa) I
started to tackle this problem adopting state-of-the-art ML Transformers. Our work aims at
identifyng the presence of merger signals from binary black holes and/or neutron stars and
provide an estimate of when the merger will occur. All this, in view of the Early Warning
strategy for electro-magnetic counterparts identification.\\
Our results were presented at the \href{https://indico.ego-gw.it/event/710/}{XIV ET Symposium (Maastrict)} and will be 
part of an upcoming publication.

\medskip

\textbf{\textcolor{black}{\hspace{0.4cm}$\circ\;$ The scientific impact of improving Virgo low frequency sensitivity}}\vspace{0.05cm}\\
\indent Virgo is expected to undergo a series of upgrades in the upcoming years. An improved low
frequency sensitivity implies higher detection rates, better sky-localization as well as the
chance to detect new astrophysical sources (like eccentric close encounters).\\
Quantifying the impact of such improvements is of vital importance in driving the roadmap of
updates and to asses the relevance of Virgo within the international network of detectors.\\
In this context, the work done during the fellowship period aimed at quantifying 
the scientific impact on eccentric burst sources by exploiting previously
developed tools and techniques.

\end{document}
